\documentclass[12pt]{article}
% Organization note: L4-LSM-semantics_packages is _mostly_ package imports, but also contains some document-level usages of those packages.
\usepackage{L4-LSM-semantics_packages}
\usepackage{L4-LSM-semantics_environments}
\usepackage{L4-LSM-semantics_macros}

%\author{Legalese (Dustin Wehr, Meng Wong, \ldots)}
\author{\href{https://legalese.com}{Legalese.com}}

\title{L4 / Linear State Machines Formal Model}
\begin{document}
\maketitle

%Section 1 has no global variables. Can easily obtain that system by taking Section 2 system and restricting it to finite-domain variables.

%In this version, all events are actions, and a special role \introx{Env}{Env} is used to model events with no role.

\noindent 

\noindent {\bf NTS and todo:}
\begin{PPI}
\item State $\to$ Section and Edge $\to$ Connection
\item Introduce a language for defining nonsingleton event schema? e.g. $\textsf{some } \vec{x}:\tau \textsf{ such that } R(\vec{x})$. {\bf No}. That can be part of L4, but in the mathematical model it would just be messy.
%\item Should I remove relievable must-next \nameforedges? {\bf No.}
%
%This was the idea for removing: Let $\evar$ be such \nameforaedge. First, change it to a may-next \nameforedge, keeping its transition guard. Its transition guard should make its enable-times disjoint from the enable-times of other relievable must-next \nameforedges at the same state with the same role. Hmm... deriving blame (which breach state to go to) is more complicated.
\end{PPI}

\tableofcontents

\section{How to Read this Document} 
{\bf Please note} that we have not yet taken the time to make this document as widely accessible as it will be eventually, because the contents are still changing frequently.

This document defines the programming language-independent mathematical model that we will use to define the semantics of our formal contracts language L4. This document also serves as the specification for the formal contract datatype that our natural language generation, formal verification, and visualization software will use.

The the next three sections contain complete formal contract model definitions, with Section \ref{globalvars} extending the model defined in Section \ref{basics}, and Section \ref{eventparams} extending the model defined in Section \ref{globalvars}. Section \ref{eventparams} is currently the most complete writeup of the L4 semantics. 

Click most terms (in \color{TermColor}this color\color{black}) to jump to their first underlined usage.

\section{Events, Time, Traces, Finite State Contracts} \label{basics}

This section defines a complete-but-limited model of contracts, called \\ \introxx{FSContracts}{simple contracts}{FSContract}{simple contract}, and also gives definitions that will be used for the full definition of \introxx{Contracts}{contracts}{Contract}{contract} in Section \ref{globalvars}.

Every \Contract specifies a \introxx{TimeUnit}{time unit}{TimeUnits}{time units}; it is the smallest unit of time that one writes constraints about or does arithmetic with. We expect it will most often be days. A \introxx{TimeStamp}{time stamp}{TimeStamps}{time stamps} is a natural number that we think of as being in units \TimeUnit.

An \introxx{Event}{event}{Events}{events} $E$ is composed of an \introxx{Action}{action}{Actions}{actions} $action_E$, a \introxx{Role}{role}{Roles}{roles} $role_E$, a \TimeStamp $timestamp_E$, and optionally some parameters (but parameters will not be introduced until Section \ref{globalvars}) $actionparams_E$. The \Actions and \Roles are fixed finite sets. In this first version of the L4 mathematical model, there is exactly one participant of each \Role.
{\bf All events are modelled as actions}, and a special \Role \introx{Env}{Env}  is used to model events that have no agent (i.e. \Role).

A \introxx{trace}{trace}{traces}{traces} is a sequence of \Events. The \TimeStamps of the \Events must be nondecreasing. Thus, within the smallest unit of time, any number of \Events can happen; however, they are always strictly ordered. The idea here is that we want \Events to be strictly ordered for simplicity and to minimize the size of the space of execution traces, but if we made the \TimeStamps strictly increasing, we would need to be working at a level of granularity for time that is at least one level smaller than the smallest unit of time that would appear in an informal version of the contract (at least when \TimeUnit = days, since contracts that use days as their minimum unit generally do not require that all \Events happen on different days).

A \Contract has a fixed finite number of \introxx{States}{sections}{State}{section}, one of which is designated the \introx{startstate}{start section}, and which includes at least the following:
\begin{LPPI}
\item \introx{fulfilled}{fulfilled}
\item \intro{$\breached{X}$} for each nonempty subset $X$ of the \Roles. There is also an \Action \intro{$\breaches{X}$} for each such $X$, and $\intro{\breach{X}{t}}$ is defined as the \Event $\tup{\breaches{X}, \Env, t}$
\end{LPPI}
Between any two events in a \trace, the \Contract is in some \introx{GlobalState}{global state} $G$ which consists of at least a \State $\state{G}$ and a \TimeStamp $\entranceTime{G}$ (in Section \ref{globalvars}, global variables will be added).

\medskip

A \Contract has a finite directed edge-labeled multigraph\footnote{By this I mean there may be multiple edges from one node to another, but they must have different labels.} which we might call its \introx{Map}{map}; the nodes are the \States, and each directed edge, which we will call a \introxx{connection}{\nameforedge}{connections}{\nameforedges}, is labeled with an \Action. The \Map is the part of the \Contract that is easy to visualize. Some notation:

\begin{LPPI}
\item For $r$ a \Role, an \intro{\depTrans{r}} is \nameforaedge whose \Role is $r$.
\item For $a$ an \Action, an \intro{\depEvent{a}} (respectively \intro{\depTrans{a}}) is an \Event (respectively \connection) whose \Action is $a$.
\item For $s$ a \State, the \intro{\sInTransitions{s}} (\intro{\sOutTransitions{s}}) are the \nameforedges (edges) coming into (going out of) $s$.
\end{LPPI}
Every \connection is one of the following three types. They will be explained in more detail in the next section.
\introxxoutside{mustntran}{must-next \nameforedge}{mustntrans}{must-next \nameforedges}
\introxxoutside{rmustntran}{relievable must-next \nameforedge}{rmustntrans}{relievable must-next \nameforedges}
\introxxoutside{mayntran}{may-next \nameforedge}{mayntrans}{may-next \nameforedges}
\begin{LPPI}
\item A \introxxinside{may-next \nameforedge} defines permitted \Events.
\item A \introxxinside{relievable must-next \nameforedge} defines the most used kind of obligated \Events. These are obligations that are relieved by the performance of a permitted {\Event} {\it by some other} agent.
\item A \introxxinside{must-next \nameforedge} defines the strongest kind of obligated \Events.
\end{LPPI}
Note that the events defined by \rmustntrans and \mustntrans are also considered permitted \Events. That completes the definition of the finite directed graph ``\Map'' view of a \Contract.

We say that \nameforaedge $\evar$ and an \Event $E = \tup{a,r,t}$ are \introx{compatible}{compatible} iff they have the same \Action $a$ and the same \Role $r$. This definition will be modified in Section \ref{globalvars} when we add \Event parameters.
\medskip



%\introx{mustntrans}{must next transitions}
%\introxx{rmustntran}{relievable must next transition}{rmustntrans}{relievable must next transitions}
%\introxx{mayntran}{may next transition}{mayntrans}{may next transitions}


%Since the environment \Env cannot breach a contract or be {\it obligated} to do anything, no $\depTrans{\Env}$ can be a \mustntran or a \rmustntran.



\medskip

Each \connection $\evar$ is also associated with a relation $\tguard{\evar}{\cdot}$ called its \introxx{TGuard}{connection guard}{TGuards}{connection guards}.\footnote{But note that in L4 programs, the relation may often be the trivial always-true relation.} For \FSContracts, it is just a relation on \TimeStamps,
and \nameforaedge\xspace $\evar$ is \introx{enabled}{enabled} upon entering a \GlobalState with \TimeStamp $t$ iff $\tguard{\evar}{t}$ is true.\footnote{Currently, LSM examples are written assuming the \TGuards of a \State s's \connections get evaluated only once upon entering the \State. It would also be reasonable to guess that they get evaluated once per \TimeUnit while the \Contract is in that state. This is not ideal.}

\medskip

Each \connection $\evar$ is also associated with a \introx{DeadlineFn}{deadline function} $\deadline{\evar}{\cdot}$, which yields a \introxx{Deadline}{deadline}{Deadlines}{deadlines}. $\deadline{\evar}{t}$ is either a \TimeStamp after $t$, or the special element \introx{nodeadline}{$\infty$}. The \Deadline for \nameforaedge is when:
%special elements \introx{discr}{discretionary} or \introx{nodeadline}{nodeadline}.
% A \Deadline is either

\begin{LPPI}
	\item an \enabled \mayntran (a kind of permission) expires\footnote{Todo: expires should probably be a defined term.}.
	\item an \enabled \mustntran (the strong form of obligation) causes a breach by $\role{\evar}$\footnote{Which recall, in this formal model means a transition to the state \breached{\{\role{\evar}\}}} if a \compatible \Event has not been performed by the deadline.
	\item an \enabled \rmustntran (the weak form of obligation) causes a possibly-joint breach by $\role{\evar}$ if a \compatible \Event has not been performed by its \Deadline\xspace {\bf and} no other permitted \Event is performed by its \Deadline.
	\end{LPPI}
For \FSContracts, a \DeadlineFn is just a function from \TimeStamps to $\TimeUnit \cup \TimeStamps$. If $d$ is such a function, and a \State is entered at \TimeStamp $t$, then:
\begin{LPPI}
\item If $d(t) \in \TimeStamps$, the deadline is $d(t)$.
	\item If $d(t) \in \TimeUnit$, the deadline is $t + d(t)$.
\end{LPPI}

The \TGuards must satisfy the following conditions, which would be statically verified in a \Contract-definition language. We give the \FSContracts definitions here, but these conditions will be used in Section \ref{globalvars} as well.
\medskip

\noindent \introx{uaoc}{unambiguous absolute obligation condition}: For every \TimeStamp t, if some \TGuard of a \mustntran evaluates to true (at $t$) then every other \TGuard evaluates to false (at $t$).
\medskip

\noindent \introx{croc}{choiceless relievable obligations condition}: For every \Role r and \TimeStamp t, if one of $r$'s \rmustntrans's \TGuards evaluates to true (at $t$) then any other \rmustntrans for $r$ evaluate to false (at $t$).
\medskip

\noindent \introx{bostgc}{breach or somewhere to go condition}: If it is possible for all the \enabled non-\Env \connections to expire simultaneously, without causing a breach (which entails that there are no enabled \mustntrans or \rmustntrans) then there must be an \depTrans{\Env} with \Deadline $\infty$.
\smallskip



%\section{Simple Contracts} \label{sc}
%
%That completes the definition of \FSContract. In the next section, we give the complete definition of how a \FSContract executes on a \trace.


\subsection{Execution for \FSContracts} \label{fscontractexec}
A \FSContract of course starts in its \startstate. Let $E_1, E_2, \dots$ be a finite or infinite \trace (recall: a sequence of \Events), as defined in Section \ref{basics}. Let $G_i$ be the \GlobalState that follows $E_i$ for each $i$.

$G_0$ is $\tup{\startstate, 0}$.
Let $i \geq 0$, and assume execution is defined up to entering $G_i$. To reduce notational clutter, let us use the aliases:
%\[ G = \tup{s,t} = G_i \qquad  G' = \tup{s',t'} = G_{i+1} \qquad A = \tup{a_A, r_A, t_A} = A_i\]
\[ G = \tup{s,t} = G_i  \qquad E = \tup{a,r,t'} = E_i \qquad  G' = \tup{s',t'} = G_{i+1}\]

{\bf Case 1}: There is some \enabled \mustntran $\evar$ in $G$. If there is any other \enabled \connection, then this \Contract (not just this \trace) violates the \uaoc, and so is invalid.\footnote{Recall that a language (tool) for \FSContracts will verify that such a thing can't happen.}
\begin{PPI}
    \item If $E$ is \compatible with $\evar$ and $E$ happens within $\evar$'s deadline, then the next state must be $\trg{\evar}$.\footnote{i.e. if $t' \leq \deadline{\evar}{t}$ then $s' = \trg{\evar}$.} This means $E$ fulfilled the obligation created by $\evar$.
        \item Otherwise, $\role{e}$ must be $r$ and $E$ must be $\breach{r}{\deadline{\evar}{t} + 1}$. %Otherwise, If
    %Otherwise, If $E$ is not \compatible with $\evar$ then $\role{\evar}$ solely breaches the contract. We specify that $E$ must be
%    \item[(c)] If $E$ is \compatible with $\evar$ and $E$ happens after $\evar$'s deadline, then $\role{\evar}$ solely breaches the contract.
\end{PPI}

{\bf Case 2}: {\bf This is correct, but obtuse -- over-concise.} There is no \enabled \mustntran in $G$. From the set of \enabled \mayntrans of $s$ {\bf and} the set of \enabled \rmustntrans in $G$, compute the deadline for each, and discard the \connections whose deadline has passed by the time $E$ happens;\footnote{i.e. discard $\evar$ if $\deadline{\evar}{t} > t'$.}  let $T_p$ be the resulting set of \connections (the $p$ is for permission). Separately, from the set of \enabled \rmustntrans in $G$, compute the deadline for each, and discard the {\connections} {\it that do not expire until after the unique minimal expiry \TimeStamp $t^*$ within the set}; let $T_o$ be the resulting set (the $o$ is for obligation), and let $R$ be $\{ \role{\evar} \, \vert \, \evar \in T_{o} \}$. Then $E$ is either:
\begin{PPI}
	\item An \Event compatible with some \connection in $T_p$.
	\item $\breach{R}{t^*}$.\footnote{Obviously not possible if $R$ is empty} In this case means that all of the \Roles whose \enabled \rmustntran expire earliest (at $t^*$) are jointly responsible for the breach.
\end{PPI}
The \bostgc ensures that one of those two cases will apply. In particular, it implies that at least one of $T_p$ or $R$ is nonempty.

%Then $A$ must be \compatible with one of the \connections in that filtered set.

%\pagebreak
%%\item If $t'$ is within $\evar$'s deadline, there are two subcases:
%\begin{LPPI}
%\item $A$'s \Role and \Action are not the same as $\evar$'s. This means $A$'s \Role performed some action while $\evar$'s role had an immediate obligation and still had time to fulfill it.
%
%\item If the next \GlobalState $G'$ is consistent with {\bf Case 2} below,\footnote{i.e. it would be a valid next \GlobalState if $\evar$ was removed from the contract.} that's fine. {\bf Importantly, \Events are really {\it action decisions}; they do not consume time. So, if Alice has an immediate obligation, it is not possible for Bob to ``starve'' Alice by repeatedly doing permitted \Events.}
%\item Otherwise, the next state $s'$ must be  $A$'s \Role's \breach state; they performed an \Event that they we're permitted to perform.
%\end{LPPI}
%
%\bigskip
%{\bf Todo: What if \Role $r$ in state $s$ has both a permitted transition and an obligated transition. I think this should be proved impossible statically. }\\
%
%Otherwise, the next state must be $\trg{\evar}$.\footnote{i.e. if $t' \leq \deadline{\evar}{t}$ then $s' = \trg{\evar}$.} This means $A$ fulfilled the obligation created by $\evar$. \\
%
%Otherwise ($t'$ is past $\evar$'s deadline), the next state {\it must} be $\evar's$ \Role's \breach state, and moreover we specify that $t'$ is exactly one \TimeUnit past the deadline.\footnote{So $G' = \tup{\breached{\{\role{\evar}\}}, \deadline{\evar}{t} + 1}$.} Thus, when a \breach occurs because of a missed deadline, the final  \TimeStamp is maximally early.\\
%
%{\bf Case 2}: There is no \enabled \mustntran. Let $T$ be the \enabled \mayntrans of $s$.\footnote{i.e. the set of $\evar$ such that $\tguard{\evar}{t}$ is true.} Compute the deadline for each \connection in $T$, and discard the \connections whose deadline has passed by the time $A$ happens.\footnote{i.e. discard $\evar$ if $\deadline{\evar}{t} > t'$.} Then $A$ must be \compatible with one of the \connections in that filtered set.







%\section{Infinite State with Global Variables, and Action Parameters} \label{globalvars}
\section{Infinite State with Global Variables} \label{globalvars}
%{\bf Note}: Action parameters are not at all necessary to utilize global variables. In fact, \Contracts with global variables and no action parameters can efficiently simulate \Contracts with global variables and action parameters.
%{\bf Todo?: Move Event Parameters to its own section. A lot can be done with infinite state systems without them.} \\
\medskip

\introxxoutside{GVars}{global vars}{GVar}{global var}
\introxoutside{GVarTypes}{gvartypes}



We introduce a set of basic datatypes $\TT$, which includes at least $\BB, \NN$, and $\ZZ$. Add to the definition of \Contract a fixed finite set of typed \introxxinside{global vars}. The \GVars are ordered, so we may describe their collective types as a single tuple $\introxinside{gvartypes} \in \TT^* $.

Add to the definition of \GlobalState an assignment of values to the \GVars. We'll call such an assignment a \introx{gvarsassign}{global vars assignment}. A particular \gvarsassign \introx{initvals}{initvals} for the values of the \GVars in the unique \startstate is required for a \Contract; thus, our a technical definition of a \Contract is fully-instantiated, without parameters. Thus, for example, there is no \Contract representation of {\it the} Y-Combinator SAFE startup financing agreement, but there is a \Contract representation of every fully instantiated signed instance of it. This is not a restriction: any \Contract-definition language, such as L4, will really be a \Contract-template definition language. Making contract parameters part of the mathematical model at this point would only serve to make the model more cumbersome.\footnote{Later, if we need to write in \LaTeX\xspace about composing contracts, we may introduce a \Contract-template mathematical model.} %   that introduces a notion of \Contract template should define that in terms of an assignment of parameters to \initvals.

%\begin{LPPI}
%\item
%\item An assignment of types ($\TT$-tuples) to the \Actions. This allows \Events to have parameters. We refer to such a type as an \introxxinside{action-parameters domain}, and the specific \AParamsDomain for \Action $a$ is $\itsAParamsDomain{a}$.
%\end{LPPI}

\introxxoutside{gvTransform}{global vars transform}{gvTransforms}{global vars transforms}

 The \Event definition receives the following generalizations:
\begin{LPPI}
\item Each \Action $a$ additionally has a \gvTransform, denoted \itsTransform{a}, which is a function from
$\GVarTypes \times \TimeStamps$ to \GVarTypes.
%$\GVarTypes \times \itsAParamsDomain{a} \times \TimeStamps$ to \GVarTypes.
\item The definition of the \TGuard of an \depTrans{a} is generalized: it may now depend on the values of the \GVars; i.e. it is now a relation on $\TimeStamps \times \GVarTypes$.
\end{LPPI}

Now a \TGuard is a relation on $\TimeStamps \times \GVarTypes$, and an $\depTrans{s}$ $\evar$ is \introx{enabledz}{enabled} upon entering a \GlobalState $\tup{s,t,\tau}$ iff $\tguard{\evar}{t,\tau}$ is true.
%\footnote{Currently, LSM examples are written assuming the \TGuards of a \State s's \connections get evaluated only once upon entering the \State. It would also be reasonable to guess that they get evaluated once per \TimeUnit while the \Contract is in that state. This is not ideal.}


The three named conditions on \TGuards are updated as follows. For every \State $s$:
\medskip

\noindent \introx{uaocz}{unique unrelievable obligation condition}: For every \GlobalState $G$ whose (local) \State is $s$, if the \TGuard of one of $s$'s \mustntrans evaluates to true (on $G$) then every other \TGuard of $s$ evaluates to false. 
\medskip

\noindent \introx{crocz}{role-unique relievable obligations condition}: For every \Role r and \GlobalState $G$ whose (local) \State is $s$, if the \TGuard of one of $s$'s \rmustntrans with \Role $r$ evaluates to true (on $G$) then the \TGuard of every other of $s$'s \rmustntrans with \Role $r$ evaluates to false.
\medskip

\noindent \introx{bostgcz}{breach or somewhere to go condition}: If it is possible for all the \enabled non-\Env \connections to expire simultaneously, without causing a breach (which entails that there are no enabled \mustntrans or \rmustntrans) then there must be an \depTrans{\Env} with \Deadline \nodeadline.


%\noindent The \uaoc is updated: Let $T$ be the set of \TGuards of the \mustntrans of some \State $s$. For any \GVars assignment $\sigma$ and \TimeStamp $t$, at most one of the \TGuards in $T$ evaluates to true. {\bf Todo}\footnote{Later will be: For any \GVars assignment $\sigma$ and \TimeStamp $t$ that makes $s$'s precondition true, at most one of the \TGuards in $T$ evaluates to true.}


\bigskip

Note (probably to move to some other section or document): it will often be the case in a \Contract-definition language that we simultaneously define an \Action $a$ and a \State \JustHappened{a} (for ``$a$ Just Happened'', to fit its literal meaning). In this case, the \sInTransitions{\JustHappened{a}} are exactly the set of $\depTranss{a}$. As a convenience, a \Contract-definition language will likely allow the \sOutTransitions{\JustHappened{a}} to depend directly on $a$'s parameters (that is, for the \TGuard
%and \ESConstructor
to depend on $a$'s parameters). This is merely a convenience because, as we will see when we define execution, one can achieve the same effect by introducing new \GVars that are only used by $a$ and $\JustHappened{a}$; $a$ uses \itsTransform{a} (recall, its \gvTransform) to save its parameter values to those new \GVars, so that the \sOutTransitions{ \JustHappened{a}} can then refer to them.












\subsection{Execution} \label{gvarsexecution}

Since Subsection \ref{fscontractexec} is short, we'll repeat essentially the entire definition of execution for \FSContracts here, rather than say how to modify it.

\medskip

Let $E_1, E_2, \dots$ be a finite or infinite \trace (recall: a sequence of \Events), as defined in Section \ref{basics}. Let $G_i$ be the \GlobalState that follows $E_i$ for each $i$. A \Contract starts in its \startstate, with initial \gvarsassign given by \initvals.

$G_0$ is $\tup{\startstate, 0, \initvals}$.
Let $i \geq 0$, and assume execution is defined up to entering $G_i$. To reduce notational clutter, let us use the aliases:
%\[ G = \tup{s,t} = G_i \qquad  G' = \tup{s',t'} = G_{i+1} \qquad A = \tup{a_A, r_A, t_A} = A_i\]
\[ G = \tup{s,t,\sigma} = G_i  \qquad E = \tup{a,r,t'} =  E_i \qquad  G' = \tup{s',t', \sigma'} = G_{i+1}\]

{\bf Case 1}: There is some \enabled \mustntran $\evar$ in $G$. If there is any other \enabled \connection, then this \Contract (not just this \trace) violates the \uaocz, and so is invalid.\footnote{Recall that a language (tool) for \Contracts will verify that such a thing can't happen.}
\begin{PPI}
    \item If $E$ is \compatible with $\evar$ and $E$ happens within $\evar$'s deadline ($t' \leq \deadline{\evar}{t}$), then the next state $s'$ must be $\trg{\evar}$, and $\sigma'$ must be $\itsTransform{a}(t,\sigma)$. This means $E$ fulfilled the obligation created by $\evar$.
    \item Otherwise, $\role{e}$ must be $r$ and $E$ must be $\breach{r}{\deadline{\evar}{t} + 1}$ and $\sigma' = \sigma$. %Otherwise, If $E$ is not \compatible with $\evar$ then $\role{\evar}$ solely breaches the contract. We specify that $E$ must be
%    \item[(c)] If $E$ is \compatible with $\evar$ and $E$ happens after $\evar$'s deadline, then $\role{\evar}$ solely breaches the contract.
\end{PPI}

{\bf Case 2}: There is no \enabled \mustntran in $G$. From the set of \enabled \mayntrans of $s$ {\bf and} the set of \enabled \rmustntrans in $G$, compute the deadline for each, and discard the \connections whose deadline has passed by the time $E$ happens\footnote{i.e. discard $\evar$ if $\deadline{\evar}{t} > t'$.};  let $T_p$ be the resulting set of \connections. From the set of \enabled \rmustntrans in $G$, compute the deadline for each, and discard the {\connections} {\it whose deadline is not the unique minimal \TimeStamp $t^*$ within that set}; let $T_o$ be the resulting set, and let $R$ be $\{ \role{\evar} \, \vert \, \evar \in T_{o} \}$. Then $E$ is either:
\begin{PPI}
	\item An \Event compatible with some \connection $e$ in $T_p$. And in this case the next state $s'$ must be $\trg{\evar}$, and $\sigma'$ must be $\itsTransform{a}(t,\sigma)$
	\item $\breach{R}{t^*}$.\footnote{Obviously not possible if $R$ is empty} This means that all of the \Roles whose \enabled \rmustntran expire earliest (at $t^*$) are jointly responsible for the breach.
\end{PPI}
The \bostgcz ensures that one of those two cases will apply. In particular, it implies that at least one of $T_p$ or $R$ is nonempty.














\section{Event Parameters and Schema} \label{eventparams}
%The feature we introduce in this section, for most uses, can be simulated efficiently by \GVars. An exception is in the case of \Events coming from

Add to the definition of \Contract an assignment of types ($\TT$-tuples) to the \Actions. This allows \Events to have parameters. We refer to such a type as an \introxx{AParamsDomain}{action-parameters domain}{AParamsDomains}{action-parameters domains}, and the specific \AParamsDomain for \Action $a$ is $\itsAParamsDomain{a}$.

\introxxoutside{EventSchema}{event schema}{EventSchemas}{event schemas}
Each \depTrans{a} $\evar$ gets assigned an \introxxinside{event schema} called $\itsESConstructor{\evar}$. An \EventSchema is a set of \Events that have the same \Action. We may think of an \EventSchema as a function from $\GVarTypes \times \TimeStamps$ to a set of $\depEvents{a}$ (for some fixed $a$). Equivalently, it is a relation on $\GVarTypes \times \TimeStamps \times \depEvents{a}$, and that is likely how it will be represented in a \Contract-definition language. %We call a set of $\depEvents{a}$ an \EventSchema. In many cases, an \EventSchema will be a singleton set. It is only for $\depEvents{\Env}$ that singleton \EventSchema allow us to express things that we couldn't otherwise; in all other cases, singleton \EventSchema can be simulated efficiently using $\GVars$. Nonetheless, we will recommend using singleton \EventSchema for events other than $\depEvents{\Env}$. Check for yourself that the definitions from the previous section do not need significant changes for singleton \EventSchema.

{\it Non-singleton} \EventSchema are most useful for an infinite or large choice of \Actions (and, in the case of $\depEvents{\Env}$, for infinite or large nondeterminism).

\EventSchema make it necessary to extend the definition of \compatible from its previous type $\Event \times \connection$ to $(\GlobalState \times \Event) \times \connection$. We say that \nameforaedge $\evar$ is \introx{compatiblez}{compatible} with  $\tup{G,E} = \tup{\tup{s,t,\sigma},\tup{a,r,t',\tau}}$ iff
$\evar$ is an \sOutTransition{s} with \Action $a$  and \Role $r$, and  $E$ is in $\itsESConstructor{\evar}(\sigma, t)$.

%\end{LPPI}

The three named conditions on \TGuards are the same as before, but we add one more. We now have both \EventSchema and \TGuards as ways of constraining when an \connection can be traversed. To reduce that redundancy we require, for every \State $s$:

\noindent \introx{nesee}{nonempty event schema for enabled \nameforedges}: For every \GlobalState $G$ whose (local) \State is $s$, any \enabled $\depTrans{s}$ $\evar$ must have $\itsESConstructor{\evar}$ nonempty (at $G$).
\medskip

\subsection{Execution} \label{schemasexecution}

Again, we elaborate the previous definition, from Subsection \ref{gvarsexecution}, of execution of a \Contract on a \trace, but we repeat all the parts from before.

\medskip

Let $E_1, E_2, \dots$ be a finite or infinite \trace. Let $G_i$ be the \GlobalState that follows $E_i$ for each $i$. A \Contract starts in its \startstate, with initial \gvarsassign given by \initvals.

$G_0$ is $\tup{\startstate, 0, \initvals}$.
Let $i \geq 0$, and assume execution is defined up to entering $G_i$. To reduce notational clutter, let us use the following aliases, and note that we have added a forth component $\epvar$ to the \Event; $\tau$ must be of type \itsAParamsDomain{a}.
%\[ G = \tup{s,t} = G_i \qquad  G' = \tup{s',t'} = G_{i+1} \qquad A = \tup{a_A, r_A, t_A} = A_i\]
\[ G = \tup{s,t,\sigma} = G_i  \qquad E = \tup{a,r,t',\tau} =  E_i \qquad  G' = \tup{s',t', \sigma'} = G_{i+1}\]

{\bf Case 1}: There is some \enabled \mustntran $\evar$ in $G$. If there is any other \enabled \connection, then this \Contract (not just this \trace) violates the \uaocz, and so is invalid.\footnote{Recall that a language (tool) for \Contracts will verify that such a thing can't happen.}
\begin{PPI}
    \item If $E$ is \compatible with $\evar$ and $E$ happens within $\evar$'s deadline ($t' \leq \deadline{\evar}{t}$), then the next state $s'$ must be $\trg{\evar}$, and $\sigma'$ must be $\itsTransform{a}(t,\sigma)$. This means $E$ fulfilled the obligation created by $\evar$.
    \item Otherwise, $\role{e}$ must be $r$ and $E$ must be $\breach{r}{\deadline{\evar}{t} + 1}$ and $\sigma' = \sigma$. %Otherwise, If $E$ is not \compatible with $\evar$ then $\role{\evar}$ solely breaches the contract. We specify that $E$ must be
%    \item[(c)] If $E$ is \compatible with $\evar$ and $E$ happens after $\evar$'s deadline, then $\role{\evar}$ solely breaches the contract.
\end{PPI}

{\bf Case 2}: There is no \enabled \mustntran in $G$. From the set of \enabled \mayntrans of $s$ {\bf and} the set of \enabled \rmustntrans in $G$, compute the deadline for each, and discard the \connections whose deadline has passed by the time $E$ happens\footnote{i.e. discard $\evar$ if $\deadline{\evar}{t} > t'$.};  let $T_p$ be the resulting set of \connections. From the set of \enabled \rmustntrans in $G$, compute the deadline for each, and discard the {\connections} {\it whose deadline is not the unique minimal \TimeStamp $t^*$ within that set}; let $T_o$ be the resulting set, and let $R$ be $\{ \role{\evar} \, \vert \, \evar \in T_{o} \}$. Then $E$ is either:
\begin{PPI}
	\item An \Event compatible with some \connection $e$ in $T_p$. And in this case the next state $s'$ must be $\trg{\evar}$, and $\sigma'$ must be $\itsTransform{a}(t,\sigma)$
	\item $\breach{R}{t^*}$.\footnote{Obviously not possible if $R$ is empty} This means that all of the \Roles whose \enabled \rmustntran expire earliest (at $t^*$) are jointly responsible for the breach.
\end{PPI}
The \bostgc ensures that one of those two cases will apply. In particular, it implies that at least one of $T_p$ or $R$ is nonempty.




%\section{Event Schema} \label{eventschema}

%\item Each \depTrans{a} additionally gets an \introxxinside{event schema} called $\itsESConstructor{a}$. This is a function from \GVarTypes\footnote{Could also easily make it depend on current \TimeStamp.} to a set of \Events all of whose \Action is $a$. We call such a set of events an \EventSchema.


\section{May-Later and Must-Later}  \label{slater}
WIP

This section does not actually change the definition of \Contract. Instead, it defines an often-useful \Contract structure that is likely to be supported with custom syntax in a \Contract-definition language.

We have so far been noncommittal about what types are available for \GVars.% and \AParamsDomains.
We will see later that the types strongly affect expressiveness. As a special case, the reader should convince themselves that any \Contract that uses only boolean (or other finite domain) types can be simulated by a \FSContract (using a much larger number of \States).

%accounted for outcome of operation to be successful, to be technically faulty, a failure commercially, a unilateral failure
%to ensure that when the contract / clause finishes, there will be no more pending obligations left (that's the condition for the conclusion of the transaction) (edited)

%Add a fourth component to the definition of \GlobalState: a finite set \EventSchemas called \FDeons. A \FDeon is a \FOb or a \FPerm. We will see soon that \FDeons are created by a new component that we add to the definition of \Transform.


%\FDeons, aside from being distinguished from \connections by a label, have the same components as \connections.


\end{document}